\usepackage{amssymb,mathrsfs,amsmath,amsfonts,amsthm,graphicx,pdfpages,MnSymbol,relsize,quiver}
\usepackage{xcolor}
\usepackage{tikz, tikz-cd}
\usepackage[section]{algorithm}
\usepackage{algorithmic}
\usepackage{wasysym}

%colors
\newcommand{\red}{{\color{red} \bullet}}
\newcommand{\green}{{\color{green} \bullet}}
\newcommand{\blue}{{\color{blue} \bullet}}
\newcommand{\define}[1]{{\bf \boldmath{#1}}}

\newcommand{\namedset}[1]{\mathbb{#1}}
\newcommand{\N}{\namedset N}

\newcommand{\X}{\mathcal{X}}
\newcommand{\Y}{\mathcal{Y}}

\newcommand{\namedcat}[1]{\mathsf{#1}}
\newcommand{\C}{\mathcal C}
\newcommand{\Cat}{\namedcat{Cat}}
\newcommand{\CMC}{\namedcat{CMC}}
\newcommand{\CMon}{\namedcat{CMon}}
\newcommand{\Mon}{\namedcat{Mon}}
\newcommand{\Set}{\namedcat{Set}}
\newcommand{\Petri}{\namedcat{Petri}}
\newcommand{\cPetri}{\namedcat{cPetri}}
\newcommand{\PSet}{\Set_{\perp}}
\newcommand{\SetPart}{\Set_{\mathrm{part}}}
\newcommand{\UNFOLD}{\mathtt{UNFOLD}}
\newcommand{\FCMon}{\mathsf{\Kl(\N)
}}
\newcommand{\Ob}{\:\mathrm{Ob}}
\newcommand{\Mor}{\:\mathrm{Mor}}
\newcommand{\Net}[1]{#1-\mathsf{Net}}
\newcommand{\Span}{\mathsf{Span}}
\newcommand{\SSpan}{\mathbb S \mathsf{pan}}
\newcommand{\CAT}{\mathsf{CAT}}
\newcommand{\im}{\mathrm{Im}\,}

\newcommand{\alphahat}{\widehat \alpha}
\newcommand{\Disp}{\mathsf{Disp}}


% \newcommand{\net}[1,2]{\begin{tikzcd}#1 \ar[r,shift left=.5ex] \ar[r,shift right=.5ex] & \N[#2] \end{tikzcd}}


\makeatletter
\newcommand*{\relrelbarsep}{.386ex}
\newcommand*{\relrelbar}{%
  \mathrel{%
    \mathpalette\@relrelbar\relrelbarsep
  }%
}
\newcommand*{\@relrelbar}[2]{%
  \raise#2\hbox to 0pt{$\m@th#1\relbar$\hss}%
  \lower#2\hbox{$\m@th#1\relbar$}%
}
\providecommand*{\rightrightarrowsfill@}{%
  \arrowfill@\relrelbar\relrelbar\rightrightarrows
}
\providecommand*{\leftleftarrowsfill@}{%
  \arrowfill@\leftleftarrows\relrelbar\relrelbar
}
\providecommand*{\xrightrightarrows}[2][]{%
  \ext@arrow 0359\rightrightarrowsfill@{#1}{#2}%
}
\providecommand*{\xleftleftarrows}[2][]{%
  \ext@arrow 3095\leftleftarrowsfill@{#1}{#2}%
}
\makeatother








%%%%%%%%%%%%%%%%%%%%%%%%%%%%%%%%%%%%%%%%%
\usepackage{subfiles}
\usepackage{stackengine}
\usepackage{mathtools}
\usepackage[utf8]{inputenc}
\usepackage{color}
\usepackage[a4paper,top=3cm,bottom=3cm,inner=3cm,outer=3cm]{geometry}

\usepackage{comment}

\usepackage{graphicx}
\usepackage{adjustbox}
\usepackage[all,2cell]{xy}\UseAllTwocells\SilentMatrices
\definecolor{darkgreen}{rgb}{0,0.45,0}
\usepackage[colorlinks,citecolor=darkgreen,urlcolor=gray,final,hyperindex,linktoc=page,pagebackref]{hyperref}
\renewcommand*{\backref}[1]{(Referred to on page #1.)}
\usepackage[capitalize]{cleveref}
\crefname{equation}{}{}
\crefname{item}{}{}
\usepackage{enumerate}



\newtheorem*{thm*}{Theorem}
\theoremstyle{remark}
\newtheorem*{rmk*}{Remark}
\newtheorem*{lem*}{Lemma}
\theoremstyle{plain}
\newtheorem*{defn*}{Definition}
\newtheorem*{cor*}{Corollary}
\theoremstyle{definition}
\newtheorem*{examples*}{Examples}
\newtheorem{prop*}{Proposition}

\theoremstyle{plain}
\newtheorem{thm}{Theorem}[section]
\theoremstyle{plain}
\newtheorem{prop}[thm]{Proposition}
\theoremstyle{remark}
\newtheorem{rmk}[thm]{Remark}
\theoremstyle{plain}
\newtheorem{lem}[thm]{Lemma}
\theoremstyle{plain}
\newtheorem{cor}[thm]{Corollary}
\theoremstyle{definition}
\newtheorem{defn}[thm]{Definition}
\theoremstyle{definition}
\newtheorem{examples}[thm]{Example}










\newcommand{\maps}{\colon}






%commenting

\newcommand{\joe}[1]{\textcolor{blue}{#1}}

\newcommand{\jade}[1]{\textcolor{purple}{#1}}
% \definecolor{purple(x11)}{rgb}{0.8, 0, 0.8}
% \def\purple{\color{purple(x11)}}
% \def\jade{\purple}




% TikZ stuff %%%%%%%%%%%%%%%%%%%%%%%%%%%%%%%%

\usetikzlibrary{arrows,positioning,fit,matrix,shapes.geometric,external}


\newcommand*\pgfdeclareanchoralias[3]{%
  \expandafter\def\csname pgf@anchor@#1@#3\expandafter\endcsname
     \expandafter{\csname pgf@anchor@#1@#2\endcsname}}

\tikzset{
    circnode/.style={
      circle, draw=red, very thin, outer sep=0.025em, minimum size=2em,
      fill=red, text centered},
    integral/.style={
      circle, draw=black, very thick, outer sep=0.025em,
      minimum size=2em, fill=blue!5, text centered},
    multiply/.style={
      circle, draw=black, very thick, outer sep=0.025em,
      minimum size=2em, fill=blue!5, text centered},
    zero/.style={
      circle, draw=black, very thick, minimum size=0.15cm, fill=black,
      inner sep=0, outer sep=0},
    bang/.style={
      circle, draw=black, very thick, minimum size=0.15cm, fill=green!10,
      inner sep=0, outer sep=0},
    delta/.style={
      regular polygon, regular polygon sides=3, minimum size=0.4cm, inner
      sep=0, outer sep=0.025em, draw=black, very thick, fill=green!10},
    codelta/.style={
      regular polygon, regular polygon sides=3, shape border rotate=180, minimum size=0.4cm,
      inner sep=0, outer sep=0.025em, draw=black, very thick, fill=green!10},
    plus/.style={
      regular polygon, regular polygon sides=3, shape border rotate=180, minimum size=0.4cm,
      inner sep = 0, outer sep=0.025em, draw=black, very thick, fill=black},
    coplus/.style={
      regular polygon, regular polygon sides=3, minimum size=0.4cm,
      inner sep = 0, outer sep=0.025em, draw=black, very thick, fill=black},
    sqnode/.style={
      regular polygon, regular polygon sides=4, minimum size=2.6em,
      draw=black, very thick, inner sep=0.2em, outer sep=0.025em,
      fill=yellow!10, text centered},
    bigcirc/.style={
      circle, draw=black, very thick, text width=1.6em, outer sep=0.025em,
      minimum height=1.6em, fill=blue!5, text centered}
}

\tikzstyle{tri}=[regular polygon,regular polygon sides=3,shape border rotate=1
80,fill=none,draw=black]

\pgfdeclareanchoralias{regular polygon}{corner 2}{left copy}
\pgfdeclareanchoralias{regular polygon}{corner 3}{right copy}
\pgfdeclareanchoralias{regular polygon}{corner 3}{addend}
\pgfdeclareanchoralias{regular polygon}{corner 2}{summand}


%TikZ


%John's Petri nets
\usepackage{tikz}
\usepackage{tikz-cd}
\usetikzlibrary{backgrounds,circuits,circuits.ee.IEC,shapes,fit,matrix}
%\tikzstyle{species}=[circle, fill=none, draw=black,scale=2]
%\tikzstyle{transition}=[rectangle, fill=none, draw=black, scale=1.5]

\def\rd{\rotatebox[origin=c]{90}{$\dashv$}} %rotate dash right
\def\ld{\rotatebox[origin=c]{-90}{$\dashv$}} %rotate dash left

\tikzstyle{simple}=[-,line width=2.000]
\tikzstyle{arrow}=[-,postaction={decorate},decoration={markings,mark=at position .5 with {\arrow{>}}},line width=1.100]
\pgfdeclarelayer{edgelayer}
\pgfdeclarelayer{nodelayer}
\pgfsetlayers{edgelayer,nodelayer,main}

\tikzstyle{none}=[inner sep=0pt]

% Petri nets
\definecolor{lblue}{rgb}{0,250,255}
\tikzstyle{species}=[circle,fill=yellow,draw=black,scale=1.15]
\tikzstyle{transition}=[rectangle,fill=lblue,draw=black,scale=1.15]
\tikzstyle{inarrow}=[->, >=stealth, shorten >=.03cm,line width=1.5]
\tikzstyle{empty}=[circle,fill=none, draw=none]
\tikzstyle{inputdot}=[circle,fill=black,draw=black, scale=.25]
\tikzstyle{inputarrow}=[->,draw=purple, shorten >=.05cm]
\tikzstyle{simple}=[-,draw=black,line width=1.000]


%Petri net package
\def\xcolorversion{2.00}
\def\xkeyvalversion{1.8}
\usetikzlibrary{arrows,shapes,decorations,automata,backgrounds,petri}
\tikzstyle{place}=[circle,thick,draw=blue!75,fill=blue!20,minimum size=6mm]
\tikzstyle{red place}=[place,draw=red!75,fill=red!20]
\tikzstyle{transition}=[rectangle,thick,draw=black!75,
  			  fill=black!20,minimum size=4mm]


\newcommand{\inv}{^{-1}}

\newcommand{\Int}{\textstyle{\int}}

\newcommand{\Kl}{\mathsf{Kl}}
\newcommand{\Khat}{\widehat K}